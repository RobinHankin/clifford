\documentclass{article}
\usepackage{xcolor}

\newcommand{\cliff}[1]{\ensuremath{C\ell\left(#1\right)}}
\begin{document}

Manuscript Number: AACA-24-00133

\section*{Rebuttal to referees' comments}

Below, I reproduce the two reviews.  I have retrofitted the PDF to
\LaTeX, and made minor changes to the typesetting, but have not
changed the sense of the comments.  Replies to the issues, and my own
observations and general comments, are in \textcolor{blue}{blue}.  I
have indicated changes to the manuscript where appropriate.


\section*{Reviewer \#2}

\textcolor{blue}{ {\bf overall response to issues raised by reviewer
    \#2}.  Work of this type is always a balance between different
  audiences, and the reviewer seems to be pointing out that the
  submission lacks one or more aspects of work.}

  
\subsection*{0.1 Summary}

Referee recommends rejection of this paper which has been submitted for publication in AACA.
There are several reasons for this negative recommendation:
\begin{itemize}
  \item The mathematical content is nonexistent to the point that it
    appears that the Author has very little appreciation for
    mathematical structure of the (universal) Clifford algebra. In
    fact, the Author does not even give a definition of Clifford
    algebra or even give a reference to the definition that he uses.

    The main objection is that the Author does not explain how he
    computes and encodes the algebra product which is critical to
    designing any package for computations with these algebras.

    In particular, from the text in Section 7 it appears that the
    Author uses identities (7.1) and (7.2) as his/her definitions of
    the contractions, which is mathematically wrong\footnote{By the
    way, the Author does not explain the syntax of such input anyway}
    whereas the left contraction (and similarly the right contraction)
    has been defined independently of the Clifford product (cf. [6,
      Sections 22.1 and 22.2]). Then, the left contraction is used,
    following the well-known Chevalley’s construction, to define the
    Clifford product via the formula

    $$
    \mathbf{x}u=\mathbf{x}\wedge u + \mathbf{x}\rfloor u,\qquad\mathbf{x}\in V,u\in\bigwedge V
    $$

and then extended by linearity and associativity to all of V [6, Page
  290] which yields an algebra isomorphic to the Clifford algebra
\cliff{Q} (or, \cliff{p, q, r})\footnote{It is strange that the Author
does not even explain the relation between a quadratic space (V, Q)
and the (universal) Clifford algebra \cliff{V, Q}, also denoted in the
literature as \cliff{Q} cf. [6].} \footnote{In fact, the Chevalley’s
construction works for a Clifford algebra \cliff{B} of any bilinear
form B and an extensive package already exists for computations with
these algebras [1, 2].} \footnote{Observe that when the Clifford
product xu is defined as in (1), which amounts to V defining the
Clifford algebra \cliff{Q} as a subalgebra of the algebra of
endomorphisms of the Grassmann algebra V, it is necessary to encode in
any package (based on this definition) both the Clifford product and
the Grassmann product. However, in the package clifford of the Author,
there is no provision for computing the wedge product} Thus, it is a
serious deficiency not to explain how the Clifford product and the
contractions are defined and computed in the package.  Of course,
there are other ways to compute the Clifford product and especially in
the algebras $\cliff{p, q} = \cliff{p, q, 0}$ generated by an
orthonormal basis such as the one based on Walsh functions cf. [6,
  Chapter 21]. The algorithm presented in [6, Section 21.3] applies to
the Clifford algebras \cliff{p, q} but it has been expanded in [3] to
include the degenerate quadratic forms for the algebras $\cliff{p, q,
  r}, r > 0$.  Nevertheless, the Author does not provide any hint
which algorithm he has used in his package.
\item According to Wikipedia [7], “R is a programming language for statistical computing and
data visualization. It has been adopted in the fields of data mining, bioinformatics, and data
analysis.” and “Packages add the capability to implement various statistical techniques such
as linear, generalized linear and nonlinear modeling, classical statistical tests, spatial analysis,
time-series analysis, and clustering.”
Thus, it is really puzzling why the Author has selected the R language as the programming
language for his/her package: it should be clear from this manuscript, that this language is
highly unsuitable for such task.

For one, the language is meant to deal with commuting numerical
input. As a consequence, the Author uses the same symbol $*$ for the
scalar product as in $5* e(1)$ and in $x * x$ (see page 1 and later)
where $x$ is an element in a Clifford algebra.  Thus, the Author
ignores the fact that the Clifford product is non-commutative in
general, and so another symbol should be used for it in the package.
For two, the syntax of the R language does not support standard
mathematical notation because it explicitly outputs terms like

\begin{verbatim}
  + 8 + 1e 1 - 10 e 2 -1e 12 ...
\end{verbatim}

instead of just

\begin{verbatim}
  8 + e 1 - 10e 2 - e 12 ...
\end{verbatim}

(see page 3 and later). Furthermore, the input is extremely awkward
like, for example,

\begin{verbatim}
  x <- clifford(list(1:3),c(2,3,7)),coeffs=3:4)
\end{verbatim}

on page 5 or

\begin{verbatim}
  (x <- clifford(list(1:3,c(1,5,7,8,10)),c(4,-10)) + 2)
\end{verbatim}

 on
page 7\footnote{By the way, the Author does not explain the syntax of such input anyway.}
\item The Author does not distinguish the unity 1 of the real field R
  from the Clifford algebra unit element. Let us denote the algebra
  unit by say $\mathbf{1}$. While we often write $1 + e_1 + 2e_{23}$
  and we do identify the number 1 with the term $\mathbf{1}$ in the
  algebra (due to the isomorphism $k\longrightarrow k\mathbf{1}$ where
  $k$ is a field) we cannot treat 1 as the scalar scalar (1) equal to
  \verb+e(2)*e(2)+ as shown on page 4 because that product really
  equals $\mathbf{1}$.  While this may appear as a minor issue when
  writing a paper, it shows a shortcoming of the package, probably
  caused by the very limited ability of the R language to deal with
  symbolic input and output, and hence inability to distinguish
  between 1 and 1. Yet, when designing any package for
  Clifford/Grassmann algebras, such distinction is crucial.
\item On page 6 in Section 6 Author computes the product $e(53)^2$ in
  a Clifford algebra of a positivedefinite inner product. The answer
  given by his/her package as \verb+scalar( 1 )+ is clearly wrong
  since, using his/her notation,

  $$
  e(53)^2=e(5)*e(3)*e(5)*e(3)=-(e(5)*e(5))*(e(3)*e(3))=-1
  $$

given that $e(5)*e(5)=e(3)*e(3)=1$ and $e(5)*e(3)=-e(3)*e(5)$.  Thus,
the question arises whether the Author’s algorithm for computing the
Clifford product miscomputes, although other products computed in the
paper, e.g., $x * x$ and $y * x$, are correct, or, perhaps there is a
typo in that paragraph on page 6.  Since the Author does not explain
his/her algorithm, nor he provides the code for his/her procedure $*$
referee cannot trace the source of this error\footnote{Author does not
explain how the compiler distinguishes between the symbol $*$ used as
the Clifford product as in $x * x$ and the same symbol $*$ used as a
scalar product as in $2 * e(2)$.}

\item The package clifford presented in the paper is extremely
  limited. It only has a few functionalities such as the computation
  of the Clifford product $*$; addition/subtraction in the algebra; a
  random generation of Clifford algebra elements by a procedure
  \verb+rcliff+ although its syntax is not explained; the left and the
  right contractions although it is not clear how they are computed;
  probably there is a procedure for computing the r-vectorpart
  $\left\langle A\right\rangle_r$ of a general Clifford algebra
  element A\footnote{ Author does not explain the notation
  $\left\langle A\right\rangle_r$} The package has none of the three
  main automorphisms built-in such grade involution; reversion; and
  conjugation [6] which is a must-to-have in any package.
\item The paper, whose purpose is to present Author’s package clifford, is very awkwardly written
since:

\begin{itemize}
 \item no code is given of any procedure;
\item  no mathematics behind any procedure is provided;
\item no comparison is made between this very simple package with any of the existing packages;
\item  no summary of the package functionalities is given;
\item  no explanation or reason is given by the Author why he/she is attempting to write yet
a new package for Clifford/Grassmann algebras in this very limiting non-symbolic R
programming language given that many other much more comprehensive packages for
Clifford/Grassmann algebras exist including a very fast package called eClifford for
the Clifford algebras C`(p, q, r). See, for example, [1, 2, 3].
\end{itemize}

\item AACA “publishes high-quality peer-reviewed research papers as well as expository and survey
articles in the area of Clifford algebras and their applications to other branches of mathematics,
physics, engineering, and related fields.” [5]. Unfortunately, the submitted manuscript is too short,
it is not well written, and it does not report on any significant development and that includes the
program clifford which is, at best, in its early stages of development. However, the R language,
in the opinion of this reviewer, is not suitable for developing a package for Clifford algebras for the
reasons summarized above.

\end{itemize}

\subsection*{0.2 Detailed comments}

In the following, referee provides some additional comments on the text.

\begin{enumerate}
  \item Abstract: This fragment “a range of different multiplication operators is provided” is confusing since only one multiplication in the Clifford algebra is provided (although not defined),
namely, the Clifford product $*$.
\item Page 2: The last two paragraphs in Section 1 are confusing. Namely, we want the vectors
$e_1,\ldots e_n$ to provide a basis in an n-dimensional real vector space V rather than just span
the space. Then, we know that the (universal) Clifford algebra \cliff{V} is finite-dimensional of
dimension $2^n$

For the basis vectors $e_i , i = 1, \ldots n$, to anti-commute in
\cliff{V}, that is, $e_i e_j = -e_i e_j$ for $i\neq j$, we must
require that the basis vectors are orthogonal.  Furthermore, $p + q$
such vectors are in fact orthonormal which, in the algebra \cliff{V},
satisfy $e_i^2 = 1$ for $1\leq i\leq p$ and $e_i^2 = -1$ for $p+1\leq
i\leq p+q$ while $e_i^2 = 0$ for $p + q + 1 \leq i\leq n$.  Let
$r=n-p-q$.  Thus, on the $(p + q)$-dimensional subspace W of V with a
basis $e_i, 1\leq i \leq p+q$ we have a non-degenerate quadratic
form Q of signature ($p, q)$ while on its orthogonal complement
$W^\bot$ of dimension r we have a null quadratic form Q0 . Thus, in
short, in display (1.1) instead of “otherwise” we should have this
range $p + q + 1 \leq i \leq n$.  Then, such Clifford algebra
\cliff{W\bot W^\bot} is denoted as \cliff{p,q,r} rather than
\cliff{p,q}. In the special case when $p = q = 0$, we have
$\cliff{0,0,r}\simeq\bigwedge V$ or $C\ell_{0,0r})$ where $\bigwedge
V$ is the Grassmann algebra.
\item  Page 2 Section 2: Author does not
  explain what a “blade” is.
\item Page 2: display at the bottom of the
page: This display has absolutely no meaning to the reader because the
Author does not explain what it means.
\item Page 3: This sentence “This
is not an issue for the maps considered here as addition and
multiplication are commutative and associative.” is puzzling since the
Clifford product is non-commutative for $n > 1$.
\item Pages 2, 3, and
later: Referee has already commented on the very awkward (and
practically unusable) input/output of the R language.
\item Page 3:
nowhere on page 3 the reader is informed what signature has been used
to compute the products $x * x$ and $z * x$. Only later on page 4 the
reader finds out that the default is the signature (p, 0, 0).
\item Page
4: the last display in Section 3 has absolutely no meaning to the
reader because the Author does not explain what it means. 1 0
when n was defined as a positive integer on 9.
\item Page 4: what is the meaning of
  $n=\left[\begin{array}{cc}1&0\\0&-1\end{array}\right]$ when $n$ was
  defined as a positive integer on page 2?
  \item Page 5: the display after the words
“then the stokes idiom for this would be: ...” has absolutely no
meaning to the reader because the Author does not explain what it
means.
\item Page 6: it was already remarked above that in signature $(p, 0,
  0), p > 0$, the square of $e_{53}$ is -1 and not 1 as shown in the
  output \verb+scalar ( 1 )+.
  \item Page 6: Formulas (7.1) and (7.2) are not stated correctly: we
    must require $s\geq r$ in (7.1) and $r \geq s$ in (7.2) with
    additional statements that $\left\langle
    A\right\rangle_r\rfloor\left\langle B\right\rangle_s=0$ and
    $\left\langle A\right\rangle_r\lfloor\left\langle
    B\right\rangle_s=0$ when $s > r$.  Thus, in the double summation
    in (7.1), the term $\left\langle\left\langle
    A\right\rangle_r\left\langle B\right\rangle_s\right\rangle_{s-r}$
    must be replaced with 0 when $r > s$ and in the the double
    summation in (7.2), the term $\left\langle\left\langle
    A\right\rangle_r\left\langle B\right\rangle_s\right\rangle_{r-s}$
    must be replaced with 0 when $s>r$. Apparently, these two special
    cases have been correctly encoded because the two examples for
    $A\rfloor B$ $A\lfloor B$ have correct outputs.
\footnote{The referee notes that the Author does not explain the
meaning of this notation: $\left\langle A\right\rangle_r$.}
\footnote{The referee also notes the awkwardness of this syntax {\tt \%\_|\%} for
the left contraction and {\tt \%|\_\%} for the right
contraction in clifford.}
\item Page 6: the
    comment at the bottom of page 6 which extends to the end of
    Section 7 on page 7 above Section 7.1, points to an irrelevant and
    awkward peculiarity of the R language and it does not deserve to
    be a part of the text: a footnote would suffice explaining why
    there is a need for the extra parentheses when computing $e_2\rfloor (e_1e_2)$.
\item Page 8: he conclusions and further work statements are
  insufficient in view of many functionalities that are still missing
  in clifford including the lack (already mentioned) of the three
  automorphims.  It should also be pointed out that the Author does not
  mention any ability on the part of the R language for handling
  matrices, let alone symbolic matrices.  Thus, it is not known whether
  it would be even possible to handle spinor or regular
  representations of the Clifford algebras \cliff{p, q, 0} in
  clifford.  Finally, the author does not explain what the stokes
  package is and why he would want to \ldots{\em include closer integration
  with the stokes package}\ldots

\end{enumerate}

\section*{References}

\begin{description}
\item{[1]} R. Ablamowicz and B. Fauser: Mathematics of CLIFFORD: A Maple Package for Clifford and
Grassmann Algebras, Adv. Applied Clifford Algebras 15 (2) (2005), 157–181.\\
\item{[2]} R. Ablamowicz and B. Fauser: CLIFFORD: A Maple Package for Clifford and Grassmann
Algebras for Maple 2020, available from R. Ablamowicz at {\tt rablamowicz@gmail.com}, 2024.\\
\item{[3]} R. Ablamowicz: eCLIFFORD: A Maple Package for Clifford and Grassmann Algebras \cliff{p, q, r}
for Maple 2020, available from R. Ablamowicz at {\tt rablamowicz@gmail.com}, 2024.\\
\item{[4]} Maple 2020.2, November 11, 2020, Copyright Maplesoft, a division of Waterloo Maple Inc.
1981-2020.\\
\item{[5]} Advances in Applied Clifford Algebras, Aims and Scope,
{\tt https://link.springer.com/journal/6}, August 2024.\\
\item{[6]} Lounesto, P., Clifford Algebras and Spinors, 2nd. ed., Cambridge University Press, 2001.
\item{[7]} R (programming language) Wikipedia, The Free Encyclopedia,
{\tt https://en.wikipedia.org/wiki/R (programming language)}, August 2024.
\end{description}


\section*{Reviewer \#3}

\end{document}
